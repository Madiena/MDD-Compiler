\thispagestyle{plain}
\vspace*{2cm}
\section{Textuelle Beschreibung}
Der Generator, der im Rahmen dieses Projektes entwickelt wurde, ist in der Lage Websiten zu generieren, welche aus mindestens einer, andernfalls aber beliebig vielen Unterseiten besteht. 

Diese Unterseiten bestehen aus jeweils einem Header, dann folgt jeweils ein Body und anschließend bestehen sie außerdem aus jeweils einem Footer.

Der Header besteht wiederum aus einem Bild und einer Navbar.

Der Body besteht aus einem oder mehreren frei wählbaren Body Elementen. Diese Elemente sind Bilder, geordnete, wie ungeordnete Listen, Formulare, Textbausteine, Links und Tabellen.

Der Footer besteht aus einem oder mehreren Links.

Die Navbar im Header setzt sich zusammen aus einer beliebigen Anordnung von Links und Dropdown-Menüs, die ihrerseits wieder einen oder mehrere Links enthalten können.
Bilder verfügen über die Angabe ihres Dateipfades.

Sortierte und unsortierte Listen bestehen aus Listenelementen.

Formulare sind stets zusammengebaut aus einem Label und einer Textarea (mehrzeilig) oder einem Label und einem Input Feld (einzeilig). Des Weiteren verfügen Formulare standardmäßig über einen Submit Button.

Textbausteine setzen sich aus einer oder mehreren Überschriften und Paragraphen zusammen, die in beliebiger Anordnung und Anzahl auftreten können, solange stets mindestens eine Überschrift oder ein Paragraph angegeben ist.

Paragraphen verfügen über den Text, der als Inhalt dargestellt werden soll.

Überschriften verfügen sowohl über Text als auch über eine numerische Angabe in welcher Größe die Überschrift angegeben wird.

Links bestehen aus einer Textangabe, die auf der Website zu lesen sein wird, sowie einer Zieladresse, wohin der Link führt.

Tabellen setzen sich zusammen aus einem Tablehead und mehreren Tablerows. Der Tablehead besteht aus Tabellenspalten, welche die einzelnen Überschriften der Spalten darstellen sollen. Die übrigen Tablerows verfügen ebenfalls über Spalten, welche allerdings den Inhalt der Tabelle darstellen. Optisch unterscheiden sich die Spaltenelemente des Tableheads von den der Tablerows dadurch, dass die Spaltenelemente des Tableheads in einer Fettschrift dargestellt sind.
