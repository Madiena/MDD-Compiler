 % Angabe zu Blattgröße, Schriftgröße, Hinzufügen des Bildverzeichnis und Quellverzeichnis als Punkt im Inhaltsverzeichnis, einseitiger Druck
\documentclass[
a4paper,12pt,listof=toc,bibliography=totoc,twoside, titlepage, headsepline,headings=optiontohead,parskip=half,
BCOR=8.25mm,         % Bindekorrektur
DIV=12,              % Satzspiegel
headsepline=on,      % Trennlinie Kolumnentitel
]{scrartcl}	
\include{settings.tex}
\newtheorem{condition}{Bedingung}[subsection]
\newtheorem{definition}{Definition}[subsection]
\theorembodyfont{\upshape}
\newmdtheoremenv[
hidealllines=true,
leftline=true,
innertopmargin=0pt,
innerbottommargin=0pt,
linewidth=4pt,
linecolor=gray!40,
innerrightmargin=0pt,
innertopmargin=-6pt,
]{example}{Beispiel}[subsection]



\title{Bachelorarbeit}
\subtitle{Erweiterung von Datalog um die Berechnung von Rängen}
\author{Madelaine Hestermann}
\mail{madelaine.hestermann@mni.thm.de}
\publishers{Betrieb:\\INFORM GmbH\\Betreuer:\\Dimitri Bohlender\\Dozent:\\Michael Elberfeld}


% % % % % % % % % % % % % %
% % % Dokument Beginn % % %
% % % % % % % % % % % % % %
\begin{document}

%% Titelseite
\begin{titlepage}
	\begin{center}
		\includegraphics[width=0.9\textwidth]{img/mni-logo}
		
		\vspace{3cm}	
		
		\huge\textbf{\sffamily Erweiterung von Datalog um die Berechnung von Rängen}
		
		\vspace{1cm}	
		
		\Large\textbf{\sffamily Bachelorarbeit}
		
		\normalsize
		\vspace{0.5cm}	
		
		zur Erlangung des akademischen Grades
		
		Bachelor of Science (B.\,Sc.)
		
		vorgelegt dem\\
		\vspace{0.2cm}
		Fachbereich Mathematik, Naturwissenschaften und Informatik\\
		\vspace{0.2cm}
		der Technischen Hochschule Mittelhessen\\
		
		von \\[1cm]	
		
		\textbf{Madelaine Tatjana Hestermann}\\ [.5cm] 
		im März 2023
	\end{center}
	\vfill
	\begin{tabular}{ll}
		Referent: & Prof. Dr. Michael Elberfeld\\ 
		Korreferent: & Dr. Dimitri Bohlender\\ 
	\end{tabular}
\end{titlepage}
\cleardoubleemptypage


%% Erklärung
\pagestyle{empty}
\begin{quote}
	\vspace*{4cm}
	
	\begin{center}
		\textbf{\Large\sffamily Eidesstattliche Erklärung}
	\end{center}
	\vspace{0.5cm}

	Hiermit versichere ich, die vorliegende Arbeit selbstständig und unter
	ausschließlicher Verwendung der angegebenen Literatur und Hilfsmittel
	erstellt zu haben.
	
	\vspace{0.3cm}
	Die Arbeit wurde bisher in gleicher oder ähnlicher Form keiner anderen
	Prü\-fungs\-be\-hör\-de vorgelegt und auch nicht veröffentlicht.
	\vspace{3em}
	
	Gießen, 28. Februar 2023
\end{quote}
\cleardoubleemptypage

%% Zusammenfassung
\pagestyle{empty}
\begin{quote}
	\begin{center}
		\textbf{\Large\sffamily Abstract}
		\vspace{0.3cm}
	\end{center}
	Deklarative Programmiersprachen sind in der Lage durch Abstraktion der Kontroll- und Datenflüsse eines Programms auf komplexe technische Details zu verzichten. Dies kann Personen ohne Programmierkenntnisse dazu befähigen entsprechenden Programmcode besser zu verstehen. Zudem sind insbesondere logische Programmiersprachen gut geeignet Applikationen zu implementieren, deren Domäne aus Regeln, wie beispielsweise gesetzlichen Bestimmungen, bestehen. Aufgrund dessen wurde für die Applikation WorkforcePlus, eine Applikation zur Organisation der Personaleinsatzplanung in Unternehmen, der INFORM GmbH eine eigene domänenspezifische Sprache entwickelt. Die Sprache Roxx ist eine solche logische Sprache und soll die genannten Vorteile des Paradigmas implementieren.
	
	\vspace{0.1cm}
	Gegenstand dieser Bachelorarbeit ist die Erweiterung eines speziellen Prädikats der Sprache Roxx, dem Partitionsprädikat. Dabei wird der bisherige Stand der theoretischen Arbeit zur Programmiersprache Datalog, der Roxx als semantische Grundlage dient, skizziert. Anschließend wird das Partitionsprädikat im Kontext von Datalog definiert. Zudem wird der Begriff der domänenspezifischen Sprache erläutert und in Zusammenhang mit Roxx erklärt. Außerdem wird eine Implementierung des Partitionsprädikats vorgestellt. Im Fokus liegen dabei die Erweiterung um die Mehrfachsortierung des Partitionsprädikats und eine neu implementierte Partitionsfunktion, welche im Rahmen dieser Bachelorarbeit entwickelt wurden. Abschließend folgt ein Ausblick inwiefern Roxx zusätzlich erweitert werden kann.
	
\end{quote}

\cleardoubleemptypage

%\begin{quote}
	
%	\begin{center}
%		\vspace*{6cm}
%			\textbf{\Large\sffamily Danksagung}
%		\vspace{0.3cm}
	
%	\end{center}
%\begin{center}
%	Danke an Frank, wegen dem alles begonnen hat.
%\end{center}
%\end{quote}
	
%	\cleardoubleemptypage
	%Inhaltsverzeichnis
	\pagenumbering{roman}
	\tableofcontents
	\clearpage
	
	%Inhalt
	\pagenumbering{arabic}
	\pagestyle{headings}
	\setcounter{page}{1} %Seitenzähler auf 1 setzen, da Inhaltsverzeichnis sonst als Seite 1 zählt
	\input{chapter/einleitung}
	\thispagestyle{plain}
\vspace*{2cm}
\section{Textuelle Beschreibung}
Der Generator, der im Rahmen dieses Projektes entwickelt wurde, ist in der Lage Websiten zu generieren, welche aus mindestens einer, andernfalls aber beliebig vielen Unterseiten besteht. 

Diese Unterseiten bestehen aus jeweils einem Header, dann folgt jeweils ein Body und anschließend bestehen sie außerdem aus jeweils einem Footer.

Der Header besteht wiederum aus einem Bild und einer Navbar.

Der Body besteht aus einem oder mehreren frei wählbaren Body Elementen. Diese Elemente sind Bilder, geordnete, wie ungeordnete Listen, Formulare, Textbausteine, Links und Tabellen.

Der Footer besteht aus einem oder mehreren Links.

Die Navbar im Header setzt sich zusammen aus einer beliebigen Anordnung von Links und Dropdown-Menüs, die ihrerseits wieder einen oder mehrere Links enthalten können.
Bilder verfügen über die Angabe ihres Dateipfades.

Sortierte und unsortierte Listen bestehen aus Listenelementen.

Formulare sind stets zusammengebaut aus einem Label und einer Textarea (mehrzeilig) oder einem Label und einem Input Feld (einzeilig). Des Weiteren verfügen Formulare standardmäßig über einen Submit Button.

Textbausteine setzen sich aus einer oder mehreren Überschriften und Paragraphen zusammen, die in beliebiger Anordnung und Anzahl auftreten können, solange stets mindestens eine Überschrift oder ein Paragraph angegeben ist.

Paragraphen verfügen über den Text, der als Inhalt dargestellt werden soll.

Überschriften verfügen sowohl über Text als auch über eine numerische Angabe in welcher Größe die Überschrift angegeben wird.

Links bestehen aus einer Textangabe, die auf der Website zu lesen sein wird, sowie einer Zieladresse, wohin der Link führt.

Tabellen setzen sich zusammen aus einem Tablehead und mehreren Tablerows. Der Tablehead besteht aus Tabellenspalten, welche die einzelnen Überschriften der Spalten darstellen sollen. Die übrigen Tablerows verfügen ebenfalls über Spalten, welche allerdings den Inhalt der Tabelle darstellen. Optisch unterscheiden sich die Spaltenelemente des Tableheads von den der Tablerows dadurch, dass die Spaltenelemente des Tableheads in einer Fettschrift dargestellt sind.

	\include{chapter/ende}
	\clearpage
	
	%Römische Seitennummerierung
	\pagenumbering{Roman}
	\setcounter{page}{1}
	
	% ================================
	% Falls nicht verwendet bitte entsprechend auskommentieren
	% ================================
	
	%Literaturverzeichnis
	\bibliographystyle{literature/bibtex/IEEEtran}	
	\bibliography{literature/Thesis.bib}
	
\end{document}